\chapter{Conclusione}

In questa tesi ho descritto la dinamica dei grani di polvere in rotazione nel mezzo interstellare, il modello di emissione anomala più accreditato. L'emissione dipende da molti parametri, legati all'ambiente interstellare e alle proprietà dei grani; ho indagato questa dipendenza usando il codice \textsc{SpDust} e simulando diversi scenari di emissione. Dallo studio è emerso che ogni parametro influenza lo spettro di emissione diversamente in ambienti differenti.

L'emissione anomala rimane un fenomeno poco conosciuto: nonostante i modelli teorici siano in continuo miglioramento, e la dinamica rotazionale sia ormai completamente descritta \parencite[20]{Ali2}, esistono tuttavia ancora molte incertezze sulle proprietà dei grani. Non solo la loro abbondanza nel mezzo interstellare, la loro grandezza e le loro proprietà chimiche sono di difficile osservazione, ma recentemente è stata anche messa in discussione l'ipotesi che i responsabili dell'emissione anomala siano gli idrocarburi aromatici policiclici \parencite{Hensley2}. Inoltre è possibile che l'emissione di dipolo elettrico sia accompagnata da emissione di dipolo magnetico \parencite{magnetic}.

Per questo motivo, sono numerosi gli esperimenti che stanno caratterizzando sempre meglio il mezzo interstellare, in modo da determinare le loro condizioni e poter quindi studiare le proprietà dei grani misurandone l'emissione.
Un naturale sbocco del lavoro di questa tesi consisterebbe appunto nel confronto con le osservazioni di emissione anomala per verificare l'efficacia della teoria. Infatti il mio lavoro potrà risultare utile per capire come esplorare lo spazio dei parametri del modello in modo da trovare il miglior fit con i dati sperimentali, un compito complesso a causa della multidimensionalità del problema.