\chapter{Introduzione}

Il termine ``\textit{emissione anomala}'' fu utilizzato per la prima volta da \textcite{L97} per indicare una componente dell'emissione Galattica nelle microonde (10--60\,GHz) osservata durante misure delle anisotropie nella radiazione cosmica di fondo (CMB).

Il mio lavoro di tesi, dopo un'introduzione di carattere storico sulla scoperta dell'emissione anomala, vuole fornire una panoramica sul più accreditato modello in grado di spiegarla e studiarne lo spettro di emissione con un codice numerico. Il modello in questione ipotizza che i grani nelle nubi di polvere abbiano un momento di dipolo e siano quindi in grado di emettere radiazione elettromagnetica se messi in rotazione (\textit{``spinning dust''}). Lo studio dello spettro è utile per approfondire la dinamica dei grani, per caratterizzare l'ambiente in cui la polvere è immersa e le proprietà della polvere stessa.

\section{Scoperta e modelli}

Il \textit{COsmic Background Explorer} \parencite[COBE,][]{Mather} fu uno dei primi satelliti lanciati per misurare la radiazione cosmica di fondo (\textit{Cosmic Microwave Background}, CMB)\footnote{il primo fu la sonda sovietica RELIKT-1; \cite{Relikt}.}, il segnale cosmologico emesso dall'universo primordiale all'epoca del disaccoppiamento tra materia e radiazione, 380.000 anni dopo il Big Bang. Uno degli strumenti, il \textit{Differential Microwave Radiometer} (DMR) aveva lo scopo di caratterizzare le anisotropie spaziali di tale segnale. Lo spettro della radiazione di fondo corrisponde a un corpo nero con $T = 2.725$\,K; le anisotropie sono piccole variazioni di questa temperatura, dell'ordine di $10^{-4}$\,K. Esse sono il risultato di fluttuazioni quantistiche nell'universo primordiale, portate poi alla scala macroscopica dal fenomeno dell'inflazione; le loro misure permettono di ricavare parametri cosmologici, come l'età dell'universo e la densità delle sue componenti, e testare teorie cosmologiche, come appunto la teoria inflazionistica.

Sette anni dopo il lancio avvenuto nel 1989, \textcite{K96} riportava una consistente correlazione nelle mappe prodotte da due diversi esperimenti a bordo di COBE: DMR e il \textit{Diffuse InfraRed Background Experiment} (DIRBE), uno strumento per mappare la distribuzione di polvere interstellare presente nella Via Lattea, rilevandone la radiazione termica nell'infrarosso (1.25--240\,$\mu$m). I risultati mostravano una forte dipendenza della radiazione misurata da DMR dall'emissione termica della polvere. Nell'articolo questa correlazione è spiegata assumendo che nelle nubi la presenza di ioni consenta agli elettroni di emettere tramite \textit{bremsstrahlung}.

L'anno successivo \textcite{L97} ottenne un risultato simile utilizzando le mappe prodotte dall'\textit{Owens Valley Radio Observatory} (OVRO) a 14.5 e 32 GHz e quelle del satellite IRAS. L'autore battezzò \textit{emissione anomala} la radiazione nella finestra delle microonde correlata alla presenza di polvere, e concluse che i segnali osservati erano compatibili sia con la radiazione di sincrotrone, sia con emissione free-free di gas ad un temperatura $T\geqslant10^6$ K. Questa condizione è necessaria poiché una temperatura minore sarebbe accompagnata da righe di emissione H$\alpha$ ordini di grandezza più intense di quelle osservate.

Poco dopo queste scoperte, un fondamentale articolo \parencite{DL98a} suggerì che la radiazione di dipolo prodotta da grani di polvere in rotazione potesse essere responsabile dell'emissione anomala. Semplici modelli per la rotazione dei grani erano stati ideati da \textcite{Erickson} e rivisitati nei decenni successivi. Nonostante l'idea fosse di molto antecedente, quella di Draine e Lazarian fu la prima descrizione teorica approfondita del fenomeno e che prendeva in considerazione un gran numero di processi che hanno luogo nelle nubi. Nell'articolo, oltre a proporre il nuovo modello, gli autori analizzarono l'ipotesi che l'emissione anomala fosse riconducibile a emissione free-free, mostrando che per osservare un bremsstrahlung di tale intensità a $T\geqslant10^6$ K sarebbe necessaria un'immissione di energia nel mezzo interstellare incompatibile con i modelli Galattici.

La polvere in rotazione è tuttora il modello più in accordo con le osservazioni di emissione anomala. Ci sono tuttavia ancora importanti punti aperti: una questione particolarmente importante è la composizione dei grani, su cui è ancora in corso un dibattito. Il modello è stato negli anni più volte aggiornato da diversi autori \parencite[vedi][]{Ali2}, in modo da tenere conto dei numerosi processi fisici che avvengono nelle nubi di polvere del mezzo interstellare. I principali modi con cui i grani possono entrare in rotazione, raccolti e analizzati da \textcite{Ali}, sono implementati nel codice \textsc{SpDust}\footnote{\textsc{SpDust} è disponibile a http://www.pha.jhu.edu/~yalihai1/spdust/spdust.html} e verranno esposti nel secondo capitolo.

\begin{figure}
\centering
\includegraphics[width=1\linewidth]{immagini/COBE2}
\caption{Mappe della sfera celeste dallo strumento DIRBE (in alto) e DMR (in basso). In entrambe le mappe è in evidenza l'emissione del disco Galattico, dove \textcite{K96} ha trovato la forte correlazione tra emissione termica della polvere e emissione ``anomala''. Nella mappa di DMR si possono notare anche le anisotropie della radiazione di fondo. A destra è rappresentato il satellite COBE. Le immagini sono disponibili a \textit{http://www.nasa.gov/}.}
\label{fig:DMR}
\end{figure}

\section{Perché studiare l'emissione anomala?}

La radiazione alle frequenze tra 10 e 1000\,GHz che osserviamo nel cielo deriva principalmente dalla CMB e da meccanismi di emissione che avvengono all'interno della nostra Galassia. I principali sono la radiazione di sincrotrone, l'emissione free-free e l'emissione della polvere: saper riconoscere queste tre componenti è dunque fondamentale nelle misure della radiazione cosmica di fondo, come suggerisce anche la storia della sua scoperta.
Mentre per le emissioni di sincrotrone e free-free è stato possibile ricavare il comportamento spettrale in funzione delle condizioni nelle quali si sviluppano (ad esempio la loro dipendenza dai campi magnetici locali, \cite{B92}), non è ancora ben conosciuta la dipendenza dell'emissione della polvere dalle caratteristiche dell'ambiente circostante. L'emissione anomala rimane quindi una delle sorgenti di disturbo nelle misure a 10--60\,GHz. 
\begin{figure}
\centering
\includegraphics[width=0.7\linewidth]{immagini/Planck}
\caption{Spettro di AME-G353.05+16.90 nella nube molecolare $\rho$ Ophiuchi West. In figura è mostrato il miglior modello che concorda con i dati. Il modello è composto da: emissione free-free (in arancione), CMB (in nero, punteggiato), emissione termica della polvere (in azzurro) e povere in rotazione. La polvere in rotazione è composta a sua volta da una componente proveniente da gas molecolare ad alta densità (in magenta) e da una proveniente da gas atomico a bassa densità (in verde). L'immagine è presa da \textit{Planck early results. XX. New light on anomalous microwave emission from spinning dust grains} (\cite{Planck}).}
\label{fig:Planck}
\end{figure}
Questa emissione ha però anche un interesse intrinseco, dal momento che la sua misura può fornire informazioni utili sulle condizioni degli ambienti interstellari osservati.
Per studiare le proprietà della polvere si fa affidamento soprattutto all'\textit{estinzione} di cui è responsabile: l'assorbimento o dispersione della radiazione che la attraversa e giunge, modificata nell'intensità e nella lunghezza d'onda, fino alla Terra. Le misure di emissione anomala costituiscono un altro modo per studiare le proprietà dei grani, che giocano un ruolo importante nella fisica e chimica del mezzo interstellare.
