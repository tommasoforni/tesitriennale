\documentclass[12pt,a4paper]{article}
\usepackage[latin1]{inputenc}
\usepackage[italian]{babel}
\usepackage{amsmath}
\usepackage{amsfonts}
\usepackage{amssymb}
\usepackage{graphicx}
\pdfminorversion=4
\title{Studio dello spettro di emissione anomala della polvere interstellare mediante il codice \textsc{SpDust}}
\author{Tommaso Forni}
\date{}
\begin{document}
	Il termine ``\textit{emissione anomala}'' si riferisce a una radiazione elettromagnetica nelle microonde di origine astrofisica la cui distribuzione � correlata con l'emissione termica nell'infrarosso della polvere interstellare. Il fenomeno � stato osservato principalmente nella Via Lattea ma � presente anche in altre galassie (vedi Hensley, Murphy e Staguhn, 2015).
	
	Questa correlazione fu notata per la prima volta da Kogut et al.(1996), confrontando le mappe del cielo ottenute da due diversi strumenti del satellite COBE: il Differential Microwave Radiometer (DMR) e il Diffuse InfraRed Background Experiment (DIRBE). L'anomalia fu inizialmente spiegata come bremsstrahlung di elettroni liberi (emissione \textit{free-free}), ma Draine e Lazarian (1998) mostrarono come questo fosse impossibile su basi energetiche, attribuendo l'emissione alla radiazione di dipolo prodotta da grani di polvere in rotazione.
		
	Comprendere l'origine dell'emissione anomala � utile poich� essa d� un contributo significativo nelle frequenze tra 10 e 60\,GHz, un intervallo importante per le misure del segnale cosmologico della radiazione cosmica di fondo. Inoltre studiandola � possibile inferire le propriet� fisiche dell'ambiente osservato e determinare le poco conosciute propriet� dei grani di polvere interstellare, una delle maggiori fonti di incertezza nello studio delle caratteristiche fisico-chimiche del mezzo interstellare. 
		
	Il modello pi� diffuso (Draine e Lazarian, 1998) ipotizza che i grani di polvere siano idrocarburi aromatici policiclici (PAHs). I grani avrebbero un momento di dipolo dovuto a difetti nella struttura molecolare, infatti i PAHs sono spesso simmetrici e non hanno tradizionalmente momento di dipolo intrinseco, e disporrebbero di molti modi per variare momento angolare, emettendo fotoni. Possono collidere con atomi o ioni del gas in cui sono immersi, possono essere messi in rotazione dall'interazione del loro momento di dipolo con ioni che si muovono nelle loro vicinanze (\textit{plasma drag}), da emissione fotoelettrica o infrarossa e persino dalla formazione di molecole H$_{2}$ sulla loro superficie. In questi processi � determinante la carica elettrica del grano, ottenuta per emissione fotoelettrica e scontri con ioni o elettroni.
	
	I modelli proposti da Draine e Lazarian sono stati aggiornati da Ali-Ha\"{\i}moud et al.(2009) che ha analizzato ogni processo di eccitazione e smorzamento della rotazione. Insieme all'articolo � distribuito un codice (\textsc{SpDust}, disponibile a http://www.pha.jhu.edu/ yalihai1/spdust/spdust.html) che calcola lo spettro di emissione in funzione di parametri ambientali (radiazione in cui immersa la polvere; densit�, temperatura e ionizzazione del gas che contiene i grani; percentuale di idrogeno molecolare) e propriet� dei grani (distribuzione della loro grandezza e forma, momento di dipolo intrinseco, efficienza del processo di formazione di H$_{2}$). Il codice deriva la distribuzione delle velocit� angolari risolvendo l'equazione di Fokker-Planck; il momento di dipolo intrinseco � un parametro libero (nell'articolo � stimato con considerazioni chimiche, ma rimane sconosciuto) e la distribuzione del volume dei grani � scelta per meglio riprodurre le osservazioni. Con questi tre elementi \textsc{SpDust} produce l'emissione prevista della nube di polvere, in particolare restituisce la potenza per unit� di angolo solido per intervallo di frequenza, in Jy cm$^2$ sr$^{-1}$.
	
	Lo scopo della mia tesi � comprendere i meccanismi dell'emissione anomala, sia studiando gli articoli che ne trattano, sia utilizzando il codice \textsc{SpDust} per studiarne lo spettro. Nella prima parte espongo il modello teorico su cui si basa il codice, che viene poi usato nella seconda parte per calcolare vari profili di emissione. Questa parte � corredata da numerosi grafici. Ho disegnato la variazione di potenza emessa in funzione di ogni parametro in diversi ambienti del mezzo interstellare, spiegandone i motivi fisici. Ho anche studiato l'interdipendenza delle condizioni, scoprendo per esempio che una piccola variazione nella ionizzazione di una nube poco illuminata (p.e. \textit{Dark Cloud}, \textit{Molecular Cloud}, regioni H I) produce grandi differenze nello spettro. Ho verificato anche che la formazione di idrogeno molecolare contribuisce poco alla distribuzione delle velocit�, cos� come il fatto che l'idrogeno nel gas sia in fase atomica o molecolare. Ho anche trovato che un'assunzione fatta nel codice riguardo la densit� numerica di elettroni porta ad un output errato quando l'emissione fotoelettrica dei grani domina sulla ionizzazione.
	
\end{document}